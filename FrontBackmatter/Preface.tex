%*****************************************
\chapter*{Preface}\label{preface}
%*****************************************

I have taught CIS 221, \textit{Digital Logic}, for Cochise College since about 2003 and enjoy working with students on this topic. From the start, I wanted students to work with labs as part of our studies and actually design circuits to complement our theoretical instruction. As I evaluated circuit design software I had three criteria:

\begin{itemize}
  \item \textbf{\ac{OER}}. It is important to me that students use software that is available free of charge and is supported by the entire web community. 
  \item \textbf{Platform}. While most of my students use a Windows-based system, some use Macintosh and it was important to me to use software that is available for both of those platforms. As a bonus, most OER software is also available for the Linux system, though I'm not aware of any of my students who are using Linux.
  \item \textbf{Simplicity}. I wanted to use software that was easy to master so students could spend their time understanding digital logic rather than learning the arcane structures of a simulation language.
\end{itemize}

I originally wrote a number of lab exercises using \textit{Logisim}, but the creator of that software, Carl Burch, announced that he would quit developing it in 2014. Because it was published as an open source project, a group of Swiss institutes started with the \textit{Logisim} software and developed a new version that integrated several new tools, like a chronogram, and released it under the name \textit{Logisim-Evolution}.

It is my hope that students will find these labs instructive and they will enhance their learning of digital logic. This lab manual is written with \LaTeX\ and published under a \href{https://creativecommons.org/publicdomain/zero/1.0/}{Creative Commons Zero} license with a goal that other instructors can modify it to meet their own needs. The source code can be found at \href{https://github.com/georgeself/CIS221_Labs}{my personal GITHUB page} and I always welcome comments that will help me improve this manual.

\bigskip
\begin{flushright}
  \textemdash  George Self
\end{flushright}


