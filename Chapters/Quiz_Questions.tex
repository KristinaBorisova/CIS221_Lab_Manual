% Note: This file contains quiz question graphics (circuit diagrams, K-Maps, etc). It is 
% used to generate a PDF file with graphic elements that can be screen captured and 
% inserted into quiz questions.

\section{Section 5.5: Simplification Using Algebra}
\subsection{Circuit Number 1}
\begin{figure}[H]
  \caption{Circuit No.1}
  \label{}	
  \myfloatalign
  \begin{tikzpicture} [circuit logic US, scale=1.00]
  % make all path lines (the node shapes) a little thicker
  \tikzstyle{every path}=[line width=0.50mm]	
  
  % Logic Gates
  \node[and  gate,inputs={nn},anchor=east] (g01) at (2.00,2.00) {};
  \node[nand gate,inputs={ni},anchor=east] (g02) at (2.00,1.00) {};
  \node[nor  gate,inputs={nn},anchor=east] (g03) at (3.50,2.50) {};
  \node[or   gate,inputs={nn},anchor=east] (g04) at (5.00,1.50) {};
  % Input nodes
  \node[circ,label={180:A}] (nA) at (0.0,2.60) {};
  \node[circ,label={180:B}] (nB) at (0.0,2.10) {};
  \node[circ,label={210:C}] (nC) at (0.0,1.90) {};
  % Connector Nodes
  \node[circ] (c01) at (0.5,2.10) {};
  \node[circ] (c02) at (0.3,1.90) {};	
  % Output nodes
  \node[circ,label={0:Y}] (y01) at (5.50,1.50) {};
  
  % Draw the lines
  \draw
  (nA) -- (g03.input 1)
  (nB) -- (c01) -- (g01.input 1)
  (c01) |- (g02.input 1)
  (nC) -- (c02) -- (g01.input 2)
  (c02) |- (g02.input 2)
  (g01.output) -- (2.25,2.00) |- (g03.input 2)
  (g03.output) -- (3.75,2.50) |- (g04.input 1)
  (g02.output) -- (3.75,1.00) |- (g04.input 2)
  (g04.output) -- (y01)
  ;		
  \end{tikzpicture}
\end{figure}

Circuit No.1 Solution:
\begin{align}
\label{}
\nonumber
(((BC)+A)'+(BC')') = Y
\end{align}

\subsection{Circuit Number 2}
\begin{figure}[H]
  \caption{Circuit No.2}
  \label{}	
  \myfloatalign
  \begin{tikzpicture} [circuit logic US, scale=1.00]
  % make all path lines (the node shapes) a little thicker
  \tikzstyle{every path}=[line width=0.50mm]	
  
  % Logic Gates
  \node[and   gate,inputs={in},anchor=east] (g01) at (2.00,2.00) {};
  \node[nor   gate,inputs={nn},anchor=east] (g02) at (2.00,1.00) {};
  \node[nand  gate,inputs={nn},anchor=east] (g03) at (3.50,1.50) {};
  \node[or    gate,inputs={nn},anchor=east] (g04) at (5.00,1.60) {};
  % Input nodes
  \node[circ,label={180:A}] (nA) at (0.0,2.50) {};
  \node[circ,label={180:B}] (nB) at (0.0,1.50) {};
  \node[circ,label={210:C}] (nC) at (0.0,0.90) {};
  % Connector Nodes
  \node[circ] (c01) at (0.5,2.50) {};
  \node[circ] (c02) at (0.5,1.50) {};	
  % Output nodes
  \node[circ,label={0:Y}] (y01) at (5.50,1.60) {};
  
  % Draw the lines
  \draw
  (nA)  -- (c01)       |- (g01.input 1)
  (c01) -- (3.75,2.50) |- (g04.input 1)
  
  (nB) -- (c02) |- (g01.input 2)
  (c02)         |- (g02.input 1)
  (nC) -- (g02.input 2)
  (g01.output) -- (2.25,2.00) |- (g03.input 1)
  (g03.output) -- (g04.input 2)
  (g02.output) -- (2.25,1.00) |- (g03.input 2)
  (g04.output) -- (y01)
  ;		
  \end{tikzpicture}
\end{figure}

Circuit No.2 Solution:
\begin{align}
\label{}
\nonumber
((A'B)(B+C)')'+ A = Y
\end{align}

\subsection{Circuit Number 3}
\begin{figure}[H]
  \caption{Circuit No.3}
  \label{}	
  \myfloatalign
  \begin{tikzpicture} [circuit logic US, scale=1.00]
  % make all path lines (the node shapes) a little thicker
  \tikzstyle{every path}=[line width=0.50mm]	
  
  % Logic Gates
  \node[or   gate,inputs={nn},anchor=east] (g01) at (2.00,3.00) {};
  \node[nand gate,inputs={nn},anchor=east] (g02) at (2.00,2.00) {};
  \node[and  gate,inputs={nn},anchor=east] (g03) at (2.00,1.00) {};
  \node[or   gate,inputs={nn},anchor=east] (g04) at (3.50,2.50) {};
  \node[or   gate,inputs={nn},anchor=east] (g05) at (3.50,1.50) {};
  \node[nand gate,inputs={nn},anchor=east] (g06) at (5.00,2.00) {};
  % Input nodes
  \node[circ,label={180:A}] (nA) at (0.0,3.10) {};
  \node[circ,label={180:B}] (nB) at (0.0,2.50) {};
  \node[circ,label={180:C}] (nC) at (0.0,1.50) {};
  \node[circ,label={180:D}] (nD) at (0.0,0.90) {};
  % Connector Nodes
  \node[circ] (c01) at (0.50,2.50) {};
  \node[circ] (c02) at (0.50,1.50) {};	
  \node[circ] (c03) at (2.30,2.00) {};	
  % Output nodes
  \node[circ,label={0:Y}] (y01) at (5.50,2.00) {};
  
  % Draw the lines
  \draw
  (nA)  -- (g01.input 1)
  (nB) -- (c01) |- (g01.input 2)
  (c01)         |- (g02.input 1)
  (nC) -- (c02) |- (g02.input 2)
  (c02)         |- (g03.input 1)
  (nD)  -- (g03.input 2)
  (g01.output) -- (2.25,3.00) |- (g04.input 1)
  (g02.output) -- (c03)       |- (g04.input 2)
  (c03) |- (g05.input 1)
  (g03.output) -- (2.25,1.00) |- (g05.input 2)
  (g04.output) -- (3.75,2.50) |- (g06.input 1)
  (g05.output) -- (3.75,1.50) |- (g06.input 2)
  (g06.output) -- (y01)
  ;		
  \end{tikzpicture}
\end{figure}

Circuit No.3 Solution:
\begin{align}
\label{}
\nonumber
(((A+B)+(BC)')((BC)'+(CD)))'
\end{align}

\subsection{Circuit Number 4}
\begin{figure}[H]
  \caption{Circuit No.4}
  \label{}	
  \myfloatalign
  \begin{tikzpicture} [circuit logic US, scale=1.00]
  % make all path lines (the node shapes) a little thicker
  \tikzstyle{every path}=[line width=0.50mm]	
  
  % Logic Gates
  \node[and  gate,inputs={nn},anchor=east] (g01) at (2.00,3.00) {};
  \node[nand gate,inputs={nn},anchor=east] (g02) at (2.00,2.00) {};
  \node[or   gate,inputs={nn},anchor=east] (g03) at (2.00,1.00) {};
  \node[or   gate,inputs={in},anchor=east] (g04) at (3.50,2.50) {};
  \node[nor  gate,inputs={nn},anchor=east] (g05) at (3.50,1.50) {};
  \node[and  gate,inputs={in},anchor=east] (g06) at (5.00,2.00) {};
  % Input nodes
  \node[circ,label={180:A}] (nA) at (0.0,3.10) {};
  \node[circ,label={180:B}] (nB) at (0.0,2.00) {};
  \node[circ,label={180:C}] (nC) at (0.0,0.90) {};
  % Connector Nodes
  \node[circ] (c01) at (0.75,3.10) {};
  \node[circ] (c02) at (0.35,2.00) {};	
  \node[circ] (c03) at (0.75,0.90) {};	
  \node[circ] (c04) at (2.30,2.00) {};	
  % Output nodes
  \node[circ,label={0:Y}] (y01) at (5.50,2.00) {};
  
  % Draw the lines
  \draw
  (nA) -- (c01) -- (g01.input 1)
  (c01)         |- (g02.input 1)
  (nB) -- (c02) |- (g01.input 2)
  (c02)         |- (g03.input 1)
  (nC) -- (c03) |- (g03.input 2)
  (c03)         |- (g02.input 2)
  (g01.output)  -- (2.25,3.00) |- (g04.input 1)
  (g02.output)  -- (c04)       |- (g04.input 2)
  (c04) |- (g05.input 1)
  (g03.output) -- (2.25,1.00) |- (g05.input 2)
  (g04.output) -- (3.75,2.50) |- (g06.input 1)
  (g05.output) -- (3.75,1.50) |- (g06.input 2)
  (g06.output) -- (y01)
  ;		
  \end{tikzpicture}
\end{figure}

Circuit No.4 Solution:
\begin{align}
\label{}
\nonumber
((AB)'+(AC)')'((AC)'+(B+C))'
\end{align}

\subsection{Circuit Number 5}
\begin{figure}[H]
  \caption{Circuit No.5}
  \label{}	
  \myfloatalign
  \begin{tikzpicture} [circuit logic US, scale=1.00]
  % make all path lines (the node shapes) a little thicker
  \tikzstyle{every path}=[line width=0.50mm]	
  
  % Logic Gates
  \node[or   gate,inputs={in},anchor=east] (g01) at (2.00,3.00) {};
  \node[nand gate,inputs={nn},anchor=east] (g02) at (2.00,2.00) {};
  \node[nor  gate,inputs={nn},anchor=east] (g03) at (2.00,1.00) {};
  \node[and  gate,inputs={in},anchor=east] (g04) at (3.50,2.50) {};
  \node[or   gate,inputs={nn},anchor=east] (g05) at (3.50,1.50) {};
  \node[or   gate,inputs={in},anchor=east] (g06) at (5.00,2.00) {};
  % Input nodes
  \node[circ,label={180:A}] (nA) at (0.0,3.10) {};
  \node[circ,label={180:B}] (nB) at (0.0,1.90) {};
  \node[circ,label={180:C}] (nC) at (0.0,0.90) {};
  % Connector Nodes
  \node[circ] (c01) at (0.35,3.10) {};
  \node[circ] (c02) at (0.35,1.90) {};	
  \node[circ] (c03) at (0.75,0.90) {};	
  \node[circ] (c04) at (2.30,2.00) {};	
  % Output nodes
  \node[circ,label={0:Y}] (y01) at (5.50,2.00) {};
  
  % Draw the lines
  \draw
  (nA) -- (c01) -- (g01.input 1)
  (c01)         |- (g02.input 1)
  (nB) -- (c02) |- (g02.input 2)
  (c02)         |- (g03.input 1)
  (nC) -- (c03) |- (g03.input 2)
  (c03)         |- (g01.input 2)
  (g01.output)  -- (2.25,3.00) |- (g04.input 1)
  (g02.output)  -- (c04)       |- (g04.input 2)
  (c04) |- (g05.input 1)
  (g03.output) -- (2.25,1.00) |- (g05.input 2)
  (g04.output) -- (3.75,2.50) |- (g06.input 1)
  (g05.output) -- (3.75,1.50) |- (g06.input 2)
  (g06.output) -- (y01)
  ;		
  \end{tikzpicture}
\end{figure}

Circuit No.5 Solution:
\begin{align}
\label{}
\nonumber
((A'+C)(AB'))'+((AB)'+(B+C)')
\end{align}




\section{Section 6: K-Maps}

\subsection{K-Map Number 1}
\begin{figure}[H]
  \myfloatalign
  \begin{tikzpicture} [circuit logic US, scale=1.00]
  % make all path lines (the node shapes) a little thicker
  \tikzstyle{every path}=[line width=0.50mm]

  %********************************************************************
  % Adjust the settings below to display the 1's and rectangles
  %********************************************************************
  % Uncomment the appropriate lines below to insert ones where needed
  % Data Row 1
   \node[] at (2.25,5.25) {\Huge $ 1 $}; % 00
   \node[] at (3.75,5.25) {\Huge $ 1 $}; % 04
  % \node[] at (5.25,5.25) {\Huge $ 1 $}; % 12
  % \node[] at (6.75,5.25) {\Huge $ 1 $}; % 08
  % Data Row 2
   \node[] at (2.25,3.75) {\Huge $ 1 $}; % 01
   \node[] at (3.75,3.75) {\Huge $ 1 $}; % 05
  % \node[] at (5.25,3.75) {\Huge $ 1 $}; % 13
  % \node[] at (6.75,3.75) {\Huge $ 1 $}; % 09
  % Data Row 3
  % \node[] at (2.25,2.25) {\Huge $ 1 $}; % 03
  % \node[] at (3.75,2.25) {\Huge $ 1 $}; % 07
  % \node[] at (5.25,2.25) {\Huge $ 1 $}; % 15
   \node[] at (6.75,2.25) {\Huge $ 1 $}; % 11
  % Data Row 4
  % \node[] at (2.25,0.75) {\Huge $ 1 $}; % 02
  % \node[] at (3.75,0.75) {\Huge $ 1 $}; % 06
  % \node[] at (5.25,0.75) {\Huge $ 1 $}; % 14
   \node[] at (6.75,0.75) {\Huge $ 1 $}; % 10
  
  % The coords for each cell - this is used to start the rectangle box
  \coordinate (cell00) at (1.50,4.50); \coordinate (cell01) at (1.50,3.00);
  \coordinate (cell02) at (1.50,0.00); \coordinate (cell03) at (1.50,1.50);
  \coordinate (cell04) at (3.00,4.50); \coordinate (cell05) at (3.00,3.00);
  \coordinate (cell06) at (3.00,0.00); \coordinate (cell07) at (3.00,1.50);
  \coordinate (cell08) at (6.00,4.50); \coordinate (cell09) at (6.00,3.00);
  \coordinate (cell10) at (6.00,0.00); \coordinate (cell11) at (6.00,1.50);
  \coordinate (cell12) at (4.50,4.50); \coordinate (cell13) at (4.50,3.00);
  \coordinate (cell14) at (4.50,0.00); \coordinate (cell15) at (4.50,1.50);
  
  % Set the ``at'' to the lower-left cell of the rectangle using the coords defined above
  % Set the minimum height/width to (number of cells) * 1.5. May have to decrease 
  % these by 0.1 to cut the rectangle just inside the cell.
%  \node [draw,
%  color=red!70!black,
%    fill=red!20!white,
%    fill opacity=0.3,
%    minimum height=1.4cm,
%    minimum width=3.0cm,
%    double,
%    rounded corners,
%    anchor=south west] at (cell15) {};
%  \node [draw,
%    color=blue!70!black,
%    fill=blue!20!white,
%    fill opacity=0.3,
%    minimum height=2.9cm,
%    minimum width=1.5cm,
%    double,
%    rounded corners,
%    anchor=south west] at (cell15) {};

  %********************************************************************
  % Shouldn't need to adjust anything below this point - this is just
  % the grid and the minterms.
  %********************************************************************	
  % Text in top-Left cell
  \node[] at (0.55,6.35) { $ \mathsf{ \mathbf{CD} } $ }; % CD
  \node[] at (1.05,7.05) { $ \mathsf{ \mathbf{AB} } $ }; % AB

  % Populate the top row header
  % In the following, the foreach lists a location/text pair
  % The the draw line draws the text at each location
  \foreach \loc/\txt in {(2.25,6.75)/{00},(3.75,6.75)/{01},(5.25,6.75)/{11},(6.75,6.75)/{10}}
    \draw \loc node{\Huge $\txt$};

  % Populate the header in column one
  \foreach \loc/\txt in {(0.75,5.25)/{00},(0.75,3.75)/{01},(0.75,2.25)/{11},(0.75,0.75)/{10}}
    \draw \loc node{\Huge $\txt$};

  % Populate the minterms
  \foreach \loc/\txt in { (2.75,4.75)/{00} , (4.25,4.75)/{04} , (5.75,4.75)/{12} , (7.25,4.75)/{08} ,
                        (2.75,3.25)/{01} , (4.25,3.25)/{05} , (5.75,3.25)/{13} , (7.25,3.25)/{09} ,
                        (2.75,1.75)/{03} , (4.25,1.75)/{07} , (5.75,1.75)/{15} , (7.25,1.75)/{11} ,
                        (2.75,0.25)/{02} , (4.25,0.25)/{06} , (5.75,0.25)/{14} , (7.25,0.25)/{10} }
    \draw \loc node{ \color{blue!90!black} \small{ $\txt$ }};

  % Draw the lines
  \draw
    % Finish drawing the grid
    [step=1.5cm,black,thin] (0,0) grid (7.5,7.5) % The Grid
    (0.0,7.5) -- (1.5,6.0) % Diagonal in the top left cell
    (1.5,6.10) -- (7.50,6.10) % Double line under top header row
    (1.40,0.0) -- (1.40,6.0) % Double line on left of header column one
  ;
  \end{tikzpicture}
  \caption{K-Map No.1}
  \label{}
\end{figure}

K-Map No.1 Minterm Expression:
\begin{align}
\label{}
\nonumber
&(A'B'C'D')+(A'B'C'D)+(A'BC'D')+(A'BC'D)+(AB'CD')+(AB'CD) \\
\nonumber
&\int(A,B,C,D) = \sum(0,1,4,5,10,11)
\end{align}

K-Map No.1 Simplification:
\begin{align}
\label{}
\nonumber
(A'C')+(AB'C)
\end{align}

\subsection{K-Map Number 2}
\begin{figure}[H]
  \myfloatalign
  \begin{tikzpicture} [circuit logic US, scale=1.00]
  % make all path lines (the node shapes) a little thicker
  \tikzstyle{every path}=[line width=0.50mm]
  
  %********************************************************************
  % Adjust the settings below to display the 1's and rectangles
  %********************************************************************
  % Uncomment the appropriate lines below to insert ones where needed
  % Data Row 1
  % \node[] at (2.25,5.25) {\Huge $ 1 $}; % 00
  % \node[] at (3.75,5.25) {\Huge $ 1 $}; % 04
   \node[] at (5.25,5.25) {\Huge $ 1 $}; % 12
  % \node[] at (6.75,5.25) {\Huge $ 1 $}; % 08
  % Data Row 2
   \node[] at (2.25,3.75) {\Huge $ 1 $}; % 01
  % \node[] at (3.75,3.75) {\Huge $ 1 $}; % 05
  % \node[] at (5.25,3.75) {\Huge $ 1 $}; % 13
   \node[] at (6.75,3.75) {\Huge $ 1 $}; % 09
  % Data Row 3
  % \node[] at (2.25,2.25) {\Huge $ 1 $}; % 03
  % \node[] at (3.75,2.25) {\Huge $ 1 $}; % 07
  % \node[] at (5.25,2.25) {\Huge $ 1 $}; % 15
  % \node[] at (6.75,2.25) {\Huge $ 1 $}; % 11
  % Data Row 4
  % \node[] at (2.25,0.75) {\Huge $ 1 $}; % 02
  % \node[] at (3.75,0.75) {\Huge $ 1 $}; % 06
   \node[] at (5.25,0.75) {\Huge $ 1 $}; % 14
  % \node[] at (6.75,0.75) {\Huge $ 1 $}; % 10
  
  % The coords for each cell - this is used to start the rectangle box
  \coordinate (cell00) at (1.50,4.50); \coordinate (cell01) at (1.50,3.00);
  \coordinate (cell02) at (1.50,0.00); \coordinate (cell03) at (1.50,1.50);
  \coordinate (cell04) at (3.00,4.50); \coordinate (cell05) at (3.00,3.00);
  \coordinate (cell06) at (3.00,0.00); \coordinate (cell07) at (3.00,1.50);
  \coordinate (cell08) at (6.00,4.50); \coordinate (cell09) at (6.00,3.00);
  \coordinate (cell10) at (6.00,0.00); \coordinate (cell11) at (6.00,1.50);
  \coordinate (cell12) at (4.50,4.50); \coordinate (cell13) at (4.50,3.00);
  \coordinate (cell14) at (4.50,0.00); \coordinate (cell15) at (4.50,1.50);
  
  % Set the ``at'' to the lower-left cell of the rectangle using the coords defined above
  % Set the minimum height/width to (number of cells) * 1.5. May have to decrease 
  % these by 0.1 to cut the rectangle just inside the cell.
  %  \node [draw,
  %  color=red!70!black,
  %    fill=red!20!white,
  %    fill opacity=0.3,
  %    minimum height=1.4cm,
  %    minimum width=3.0cm,
  %    double,
  %    rounded corners,
  %    anchor=south west] at (cell15) {};
  %  \node [draw,
  %    color=blue!70!black,
  %    fill=blue!20!white,
  %    fill opacity=0.3,
  %    minimum height=2.9cm,
  %    minimum width=1.5cm,
  %    double,
  %    rounded corners,
  %    anchor=south west] at (cell15) {};
  
  %********************************************************************
  % Shouldn't need to adjust anything below this point - this is just
  % the grid and the minterms.
  %********************************************************************	
  % Text in top-Left cell
  \node[] at (0.55,6.35) { $ \mathsf{ \mathbf{CD} } $ }; % CD
  \node[] at (1.05,7.05) { $ \mathsf{ \mathbf{AB} } $ }; % AB
  
  % Populate the top row header
  % In the following, the foreach lists a location/text pair
  % The the draw line draws the text at each location
  \foreach \loc/\txt in {(2.25,6.75)/{00},(3.75,6.75)/{01},(5.25,6.75)/{11},(6.75,6.75)/{10}}
  \draw \loc node{\Huge $\txt$};
  
  % Populate the header in column one
  \foreach \loc/\txt in {(0.75,5.25)/{00},(0.75,3.75)/{01},(0.75,2.25)/{11},(0.75,0.75)/{10}}
  \draw \loc node{\Huge $\txt$};
  
  % Populate the minterms
  \foreach \loc/\txt in { (2.75,4.75)/{00} , (4.25,4.75)/{04} , (5.75,4.75)/{12} , (7.25,4.75)/{08} ,
    (2.75,3.25)/{01} , (4.25,3.25)/{05} , (5.75,3.25)/{13} , (7.25,3.25)/{09} ,
    (2.75,1.75)/{03} , (4.25,1.75)/{07} , (5.75,1.75)/{15} , (7.25,1.75)/{11} ,
    (2.75,0.25)/{02} , (4.25,0.25)/{06} , (5.75,0.25)/{14} , (7.25,0.25)/{10} }
  \draw \loc node{ \color{blue!90!black} \small{ $\txt$ }};
  
  % Draw the lines
  \draw
  % Finish drawing the grid
  [step=1.5cm,black,thin] (0,0) grid (7.5,7.5) % The Grid
  (0.0,7.5) -- (1.5,6.0) % Diagonal in the top left cell
  (1.5,6.10) -- (7.50,6.10) % Double line under top header row
  (1.40,0.0) -- (1.40,6.0) % Double line on left of header column one
  ;
  \end{tikzpicture}
  \caption{K-Map No.2}
  \label{}
\end{figure}

K-Map No.2 Minterm Expression:
\begin{align}
\label{}
\nonumber
&(A'B'C'D)+(AB'C'D)+(ABC'D')+(ABCD') \\
\nonumber
&\int(A,B,C,D) = \sum(1,9,12,14)
\end{align}

K-Map No.2 Simplified:
\begin{align}
\label{}
\nonumber
(ABD)+(B'C'D)
\end{align}

\subsection{K-Map Number 3}
\begin{figure}[H]
  \myfloatalign
  \begin{tikzpicture} [circuit logic US, scale=1.00]
  % make all path lines (the node shapes) a little thicker
  \tikzstyle{every path}=[line width=0.50mm]
  
  %********************************************************************
  % Adjust the settings below to display the 1's and rectangles
  %********************************************************************
  % Uncomment the appropriate lines below to insert ones where needed
  % Data Row 1
  % \node[] at (2.25,5.25) {\Huge $ 1 $}; % 00
  % \node[] at (3.75,5.25) {\Huge $ 1 $}; % 04
  % \node[] at (5.25,5.25) {\Huge $ 1 $}; % 12
  % \node[] at (6.75,5.25) {\Huge $ 1 $}; % 08
  % Data Row 2
   \node[] at (2.25,3.75) {\Huge $ 1 $}; % 01
   \node[] at (3.75,3.75) {\Huge $ 1 $}; % 05
   \node[] at (5.25,3.75) {\Huge $ 1 $}; % 13
   \node[] at (6.75,3.75) {\Huge $ 1 $}; % 09
  % Data Row 3
  % \node[] at (2.25,2.25) {\Huge $ 1 $}; % 03
  % \node[] at (3.75,2.25) {\Huge $ 1 $}; % 07
  % \node[] at (5.25,2.25) {\Huge $ 1 $}; % 15
  % \node[] at (6.75,2.25) {\Huge $ 1 $}; % 11
  % Data Row 4
   \node[] at (2.25,0.75) {\Huge $ 1 $}; % 02
   \node[] at (3.75,0.75) {\Huge $ 1 $}; % 06
   \node[] at (5.25,0.75) {\Huge $ 1 $}; % 14
   \node[] at (6.75,0.75) {\Huge $ 1 $}; % 10
  
  % The coords for each cell - this is used to start the rectangle box
  \coordinate (cell00) at (1.50,4.50); \coordinate (cell01) at (1.50,3.00);
  \coordinate (cell02) at (1.50,0.00); \coordinate (cell03) at (1.50,1.50);
  \coordinate (cell04) at (3.00,4.50); \coordinate (cell05) at (3.00,3.00);
  \coordinate (cell06) at (3.00,0.00); \coordinate (cell07) at (3.00,1.50);
  \coordinate (cell08) at (6.00,4.50); \coordinate (cell09) at (6.00,3.00);
  \coordinate (cell10) at (6.00,0.00); \coordinate (cell11) at (6.00,1.50);
  \coordinate (cell12) at (4.50,4.50); \coordinate (cell13) at (4.50,3.00);
  \coordinate (cell14) at (4.50,0.00); \coordinate (cell15) at (4.50,1.50);
  
  % Set the ``at'' to the lower-left cell of the rectangle using the coords defined above
  % Set the minimum height/width to (number of cells) * 1.5. May have to decrease 
  % these by 0.1 to cut the rectangle just inside the cell.
  %  \node [draw,
  %  color=red!70!black,
  %    fill=red!20!white,
  %    fill opacity=0.3,
  %    minimum height=1.4cm,
  %    minimum width=3.0cm,
  %    double,
  %    rounded corners,
  %    anchor=south west] at (cell15) {};
  %  \node [draw,
  %    color=blue!70!black,
  %    fill=blue!20!white,
  %    fill opacity=0.3,
  %    minimum height=2.9cm,
  %    minimum width=1.5cm,
  %    double,
  %    rounded corners,
  %    anchor=south west] at (cell15) {};
  
  %********************************************************************
  % Shouldn't need to adjust anything below this point - this is just
  % the grid and the minterms.
  %********************************************************************	
  % Text in top-Left cell
  \node[] at (0.55,6.35) { $ \mathsf{ \mathbf{CD} } $ }; % CD
  \node[] at (1.05,7.05) { $ \mathsf{ \mathbf{AB} } $ }; % AB
  
  % Populate the top row header
  % In the following, the foreach lists a location/text pair
  % The the draw line draws the text at each location
  \foreach \loc/\txt in {(2.25,6.75)/{00},(3.75,6.75)/{01},(5.25,6.75)/{11},(6.75,6.75)/{10}}
  \draw \loc node{\Huge $\txt$};
  
  % Populate the header in column one
  \foreach \loc/\txt in {(0.75,5.25)/{00},(0.75,3.75)/{01},(0.75,2.25)/{11},(0.75,0.75)/{10}}
  \draw \loc node{\Huge $\txt$};
  
  % Populate the minterms
  \foreach \loc/\txt in { (2.75,4.75)/{00} , (4.25,4.75)/{04} , (5.75,4.75)/{12} , (7.25,4.75)/{08} ,
    (2.75,3.25)/{01} , (4.25,3.25)/{05} , (5.75,3.25)/{13} , (7.25,3.25)/{09} ,
    (2.75,1.75)/{03} , (4.25,1.75)/{07} , (5.75,1.75)/{15} , (7.25,1.75)/{11} ,
    (2.75,0.25)/{02} , (4.25,0.25)/{06} , (5.75,0.25)/{14} , (7.25,0.25)/{10} }
  \draw \loc node{ \color{blue!90!black} \small{ $\txt$ }};
  
  % Draw the lines
  \draw
  % Finish drawing the grid
  [step=1.5cm,black,thin] (0,0) grid (7.5,7.5) % The Grid
  (0.0,7.5) -- (1.5,6.0) % Diagonal in the top left cell
  (1.5,6.10) -- (7.50,6.10) % Double line under top header row
  (1.40,0.0) -- (1.40,6.0) % Double line on left of header column one
  ;
  \end{tikzpicture}
  \caption{K-Map No.3}
  \label{}
\end{figure}

K-Map No.3 Minterm Expression:
\begin{align}
\label{}
\nonumber
&(A'B'C'D)+(A'B'CD')+(A'BC'D)+(A'BCD')+(AB'C'D)+(AB'CD')+(ABC'D)+(ABCD') \\
\nonumber
&\int(A,B,C,D) = \sum(1,2,5,6,9,10,13,14)
\end{align}

K-Map No.3 Simplification:
\begin{align}
\nonumber
C \oplus D
\end{align}

\subsection{K-Map Number 4}
\begin{figure}[H]
  \myfloatalign
  \begin{tikzpicture} [circuit logic US, scale=1.00]
  % make all path lines (the node shapes) a little thicker
  \tikzstyle{every path}=[line width=0.50mm]
  
  %********************************************************************
  % Adjust the settings below to display the 1's and rectangles
  %********************************************************************
  % Uncomment the appropriate lines below to insert ones where needed
  % Data Row 1
   \node[] at (2.25,5.25) {\Huge $ 1 $}; % 00
  % \node[] at (3.75,5.25) {\Huge $ 1 $}; % 04
   \node[] at (5.25,5.25) {\Huge $ 1 $}; % 12
  % \node[] at (6.75,5.25) {\Huge $ 1 $}; % 08
  % Data Row 2
   \node[] at (2.25,3.75) {\Huge $ 1 $}; % 01
  % \node[] at (3.75,3.75) {\Huge $ 1 $}; % 05
   \node[] at (5.25,3.75) {\Huge $ 1 $}; % 13
  % \node[] at (6.75,3.75) {\Huge $ 1 $}; % 09
  % Data Row 3
   \node[] at (2.25,2.25) {\Huge $ 1 $}; % 03
  % \node[] at (3.75,2.25) {\Huge $ 1 $}; % 07
   \node[] at (5.25,2.25) {\Huge $ 1 $}; % 15
  % \node[] at (6.75,2.25) {\Huge $ 1 $}; % 11
  % Data Row 4
   \node[] at (2.25,0.75) {\Huge $ 1 $}; % 02
  % \node[] at (3.75,0.75) {\Huge $ 1 $}; % 06
   \node[] at (5.25,0.75) {\Huge $ 1 $}; % 14
  % \node[] at (6.75,0.75) {\Huge $ 1 $}; % 10
  
  % The coords for each cell - this is used to start the rectangle box
  \coordinate (cell00) at (1.50,4.50); \coordinate (cell01) at (1.50,3.00);
  \coordinate (cell02) at (1.50,0.00); \coordinate (cell03) at (1.50,1.50);
  \coordinate (cell04) at (3.00,4.50); \coordinate (cell05) at (3.00,3.00);
  \coordinate (cell06) at (3.00,0.00); \coordinate (cell07) at (3.00,1.50);
  \coordinate (cell08) at (6.00,4.50); \coordinate (cell09) at (6.00,3.00);
  \coordinate (cell10) at (6.00,0.00); \coordinate (cell11) at (6.00,1.50);
  \coordinate (cell12) at (4.50,4.50); \coordinate (cell13) at (4.50,3.00);
  \coordinate (cell14) at (4.50,0.00); \coordinate (cell15) at (4.50,1.50);
  
  % Set the ``at'' to the lower-left cell of the rectangle using the coords defined above
  % Set the minimum height/width to (number of cells) * 1.5. May have to decrease 
  % these by 0.1 to cut the rectangle just inside the cell.
  %  \node [draw,
  %  color=red!70!black,
  %    fill=red!20!white,
  %    fill opacity=0.3,
  %    minimum height=1.4cm,
  %    minimum width=3.0cm,
  %    double,
  %    rounded corners,
  %    anchor=south west] at (cell15) {};
  %  \node [draw,
  %    color=blue!70!black,
  %    fill=blue!20!white,
  %    fill opacity=0.3,
  %    minimum height=2.9cm,
  %    minimum width=1.5cm,
  %    double,
  %    rounded corners,
  %    anchor=south west] at (cell15) {};
  
  %********************************************************************
  % Shouldn't need to adjust anything below this point - this is just
  % the grid and the minterms.
  %********************************************************************	
  % Text in top-Left cell
  \node[] at (0.55,6.35) { $ \mathsf{ \mathbf{CD} } $ }; % CD
  \node[] at (1.05,7.05) { $ \mathsf{ \mathbf{AB} } $ }; % AB
  
  % Populate the top row header
  % In the following, the foreach lists a location/text pair
  % The the draw line draws the text at each location
  \foreach \loc/\txt in {(2.25,6.75)/{00},(3.75,6.75)/{01},(5.25,6.75)/{11},(6.75,6.75)/{10}}
  \draw \loc node{\Huge $\txt$};
  
  % Populate the header in column one
  \foreach \loc/\txt in {(0.75,5.25)/{00},(0.75,3.75)/{01},(0.75,2.25)/{11},(0.75,0.75)/{10}}
  \draw \loc node{\Huge $\txt$};
  
  % Populate the minterms
  \foreach \loc/\txt in { (2.75,4.75)/{00} , (4.25,4.75)/{04} , (5.75,4.75)/{12} , (7.25,4.75)/{08} ,
    (2.75,3.25)/{01} , (4.25,3.25)/{05} , (5.75,3.25)/{13} , (7.25,3.25)/{09} ,
    (2.75,1.75)/{03} , (4.25,1.75)/{07} , (5.75,1.75)/{15} , (7.25,1.75)/{11} ,
    (2.75,0.25)/{02} , (4.25,0.25)/{06} , (5.75,0.25)/{14} , (7.25,0.25)/{10} }
  \draw \loc node{ \color{blue!90!black} \small{ $\txt$ }};
  
  % Draw the lines
  \draw
  % Finish drawing the grid
  [step=1.5cm,black,thin] (0,0) grid (7.5,7.5) % The Grid
  (0.0,7.5) -- (1.5,6.0) % Diagonal in the top left cell
  (1.5,6.10) -- (7.50,6.10) % Double line under top header row
  (1.40,0.0) -- (1.40,6.0) % Double line on left of header column one
  ;
  \end{tikzpicture}
  \caption{K-Map No.4}
  \label{}
\end{figure}

K-Map No.4 Minterm Expression:
\begin{align}
\label{}
\nonumber
&(A'B'C'D')+(A'B'C'D)+(A'B'C'D)+(A'B'CD)+(ABC'D')+(ABC'D)+(ABCD')+(ABCD) \\
\nonumber
&\int(A,B,C,D) = \sum(0,1,2,3,12,13,14,15)
\end{align}

K-Map No.4 Simplification:
\begin{align}
\label{}
\nonumber
A \odot B
\end{align}

\subsection{K-Map Number 5}
\begin{figure}[H]
  \myfloatalign
  \begin{tikzpicture} [circuit logic US, scale=1.00]
  % make all path lines (the node shapes) a little thicker
  \tikzstyle{every path}=[line width=0.50mm]
  
  %********************************************************************
  % Adjust the settings below to display the 1's and rectangles
  %********************************************************************
  % Uncomment the appropriate lines below to insert ones where needed
  % Data Row 1
  % \node[] at (2.25,5.25) {\Huge $ 1 $}; % 00
   \node[] at (3.75,5.25) {\Huge $ 1 $}; % 04
  % \node[] at (5.25,5.25) {\Huge $ 1 $}; % 12
  % \node[] at (6.75,5.25) {\Huge $ 1 $}; % 08
  % Data Row 2
  % \node[] at (2.25,3.75) {\Huge $ 1 $}; % 01
   \node[] at (3.75,3.75) {\Huge $ 1 $}; % 05
   \node[] at (5.25,3.75) {\Huge $ 1 $}; % 13
   \node[] at (6.75,3.75) {\Huge $ 1 $}; % 09
  % Data Row 3
  % \node[] at (2.25,2.25) {\Huge $ 1 $}; % 03
  % \node[] at (3.75,2.25) {\Huge $ 1 $}; % 07
  % \node[] at (5.25,2.25) {\Huge $ 1 $}; % 15
  % \node[] at (6.75,2.25) {\Huge $ 1 $}; % 11
  % Data Row 4
   \node[] at (2.25,0.75) {\Huge $ 1 $}; % 02
  % \node[] at (3.75,0.75) {\Huge $ 1 $}; % 06
  % \node[] at (5.25,0.75) {\Huge $ 1 $}; % 14
  % \node[] at (6.75,0.75) {\Huge $ 1 $}; % 10
  
  % The coords for each cell - this is used to start the rectangle box
  \coordinate (cell00) at (1.50,4.50); \coordinate (cell01) at (1.50,3.00);
  \coordinate (cell02) at (1.50,0.00); \coordinate (cell03) at (1.50,1.50);
  \coordinate (cell04) at (3.00,4.50); \coordinate (cell05) at (3.00,3.00);
  \coordinate (cell06) at (3.00,0.00); \coordinate (cell07) at (3.00,1.50);
  \coordinate (cell08) at (6.00,4.50); \coordinate (cell09) at (6.00,3.00);
  \coordinate (cell10) at (6.00,0.00); \coordinate (cell11) at (6.00,1.50);
  \coordinate (cell12) at (4.50,4.50); \coordinate (cell13) at (4.50,3.00);
  \coordinate (cell14) at (4.50,0.00); \coordinate (cell15) at (4.50,1.50);
  
  % Set the ``at'' to the lower-left cell of the rectangle using the coords defined above
  % Set the minimum height/width to (number of cells) * 1.5. May have to decrease 
  % these by 0.1 to cut the rectangle just inside the cell.
  %  \node [draw,
  %  color=red!70!black,
  %    fill=red!20!white,
  %    fill opacity=0.3,
  %    minimum height=1.4cm,
  %    minimum width=3.0cm,
  %    double,
  %    rounded corners,
  %    anchor=south west] at (cell15) {};
  %  \node [draw,
  %    color=blue!70!black,
  %    fill=blue!20!white,
  %    fill opacity=0.3,
  %    minimum height=2.9cm,
  %    minimum width=1.5cm,
  %    double,
  %    rounded corners,
  %    anchor=south west] at (cell15) {};
  
  %********************************************************************
  % Shouldn't need to adjust anything below this point - this is just
  % the grid and the minterms.
  %********************************************************************	
  % Text in top-Left cell
  \node[] at (0.55,6.35) { $ \mathsf{ \mathbf{CD} } $ }; % CD
  \node[] at (1.05,7.05) { $ \mathsf{ \mathbf{AB} } $ }; % AB
  
  % Populate the top row header
  % In the following, the foreach lists a location/text pair
  % The the draw line draws the text at each location
  \foreach \loc/\txt in {(2.25,6.75)/{00},(3.75,6.75)/{01},(5.25,6.75)/{11},(6.75,6.75)/{10}}
  \draw \loc node{\Huge $\txt$};
  
  % Populate the header in column one
  \foreach \loc/\txt in {(0.75,5.25)/{00},(0.75,3.75)/{01},(0.75,2.25)/{11},(0.75,0.75)/{10}}
  \draw \loc node{\Huge $\txt$};
  
  % Populate the minterms
  \foreach \loc/\txt in { (2.75,4.75)/{00} , (4.25,4.75)/{04} , (5.75,4.75)/{12} , (7.25,4.75)/{08} ,
    (2.75,3.25)/{01} , (4.25,3.25)/{05} , (5.75,3.25)/{13} , (7.25,3.25)/{09} ,
    (2.75,1.75)/{03} , (4.25,1.75)/{07} , (5.75,1.75)/{15} , (7.25,1.75)/{11} ,
    (2.75,0.25)/{02} , (4.25,0.25)/{06} , (5.75,0.25)/{14} , (7.25,0.25)/{10} }
  \draw \loc node{ \color{blue!90!black} \small{ $\txt$ }};
  
  % Draw the lines
  \draw
  % Finish drawing the grid
  [step=1.5cm,black,thin] (0,0) grid (7.5,7.5) % The Grid
  (0.0,7.5) -- (1.5,6.0) % Diagonal in the top left cell
  (1.5,6.10) -- (7.50,6.10) % Double line under top header row
  (1.40,0.0) -- (1.40,6.0) % Double line on left of header column one
  ;
  \end{tikzpicture}
  \caption{K-Map No.5}
  \label{}
\end{figure}

K-Map No.5 Minterm Expression:
\begin{align}
\label{}
\nonumber
&(A'B'CD')+(A'BC'D')+(A'BC'D)+(ABC'D)+(AB'C'D) \\
\nonumber
&\int(A,B,C,D) = \sum(2,4,5,9,13)
\end{align}

K-Map No.5 Simplification:
\begin{align}
\label{}
\nonumber
(A'BC')+(AC'D)+(A'B'CD')
\end{align}

\subsection{K-Map Number 6}
\begin{figure}[H]
  \myfloatalign
  \begin{tikzpicture} [circuit logic US, scale=1.00]
  % make all path lines (the node shapes) a little thicker
  \tikzstyle{every path}=[line width=0.50mm]
  
  %********************************************************************
  % Adjust the settings below to display the 1's and rectangles
  %********************************************************************
  % Uncomment the appropriate lines below to insert ones where needed
  % Data Row 1
  % \node[] at (2.25,5.25) {\Huge $ 1 $}; % 00
  % \node[] at (3.75,5.25) {\Huge $ 1 $}; % 04
  % \node[] at (5.25,5.25) {\Huge $ 1 $}; % 12
   \node[] at (6.75,5.25) {\Huge $ 1 $}; % 08
  % Data Row 2
  % \node[] at (2.25,3.75) {\Huge $ 1 $}; % 01
   \node[] at (3.75,3.75) {\Huge $ 1 $}; % 05
   \node[] at (5.25,3.75) {\Huge $ 1 $}; % 13
   \node[] at (6.75,3.75) {\Huge $ 1 $}; % 09
  % Data Row 3
  % \node[] at (2.25,2.25) {\Huge $ 1 $}; % 03
  % \node[] at (3.75,2.25) {\Huge $ 1 $}; % 07
  % \node[] at (5.25,2.25) {\Huge $ 1 $}; % 15
  % \node[] at (6.75,2.25) {\Huge $ 1 $}; % 11
  % Data Row 4
  % \node[] at (2.25,0.75) {\Huge $ 1 $}; % 02
  % \node[] at (3.75,0.75) {\Huge $ 1 $}; % 06
  % \node[] at (5.25,0.75) {\Huge $ 1 $}; % 14
   \node[] at (6.75,0.75) {\Huge $ 1 $}; % 10
  
  % The coords for each cell - this is used to start the rectangle box
  \coordinate (cell00) at (1.50,4.50); \coordinate (cell01) at (1.50,3.00);
  \coordinate (cell02) at (1.50,0.00); \coordinate (cell03) at (1.50,1.50);
  \coordinate (cell04) at (3.00,4.50); \coordinate (cell05) at (3.00,3.00);
  \coordinate (cell06) at (3.00,0.00); \coordinate (cell07) at (3.00,1.50);
  \coordinate (cell08) at (6.00,4.50); \coordinate (cell09) at (6.00,3.00);
  \coordinate (cell10) at (6.00,0.00); \coordinate (cell11) at (6.00,1.50);
  \coordinate (cell12) at (4.50,4.50); \coordinate (cell13) at (4.50,3.00);
  \coordinate (cell14) at (4.50,0.00); \coordinate (cell15) at (4.50,1.50);
  
  % Set the ``at'' to the lower-left cell of the rectangle using the coords defined above
  % Set the minimum height/width to (number of cells) * 1.5. May have to decrease 
  % these by 0.1 to cut the rectangle just inside the cell.
  %  \node [draw,
  %  color=red!70!black,
  %    fill=red!20!white,
  %    fill opacity=0.3,
  %    minimum height=1.4cm,
  %    minimum width=3.0cm,
  %    double,
  %    rounded corners,
  %    anchor=south west] at (cell15) {};
  %  \node [draw,
  %    color=blue!70!black,
  %    fill=blue!20!white,
  %    fill opacity=0.3,
  %    minimum height=2.9cm,
  %    minimum width=1.5cm,
  %    double,
  %    rounded corners,
  %    anchor=south west] at (cell15) {};
  
  %********************************************************************
  % Shouldn't need to adjust anything below this point - this is just
  % the grid and the minterms.
  %********************************************************************	
  % Text in top-Left cell
  \node[] at (0.55,6.35) { $ \mathsf{ \mathbf{CD} } $ }; % CD
  \node[] at (1.05,7.05) { $ \mathsf{ \mathbf{AB} } $ }; % AB
  
  % Populate the top row header
  % In the following, the foreach lists a location/text pair
  % The the draw line draws the text at each location
  \foreach \loc/\txt in {(2.25,6.75)/{00},(3.75,6.75)/{01},(5.25,6.75)/{11},(6.75,6.75)/{10}}
  \draw \loc node{\Huge $\txt$};
  
  % Populate the header in column one
  \foreach \loc/\txt in {(0.75,5.25)/{00},(0.75,3.75)/{01},(0.75,2.25)/{11},(0.75,0.75)/{10}}
  \draw \loc node{\Huge $\txt$};
  
  % Populate the minterms
  \foreach \loc/\txt in { (2.75,4.75)/{00} , (4.25,4.75)/{04} , (5.75,4.75)/{12} , (7.25,4.75)/{08} ,
    (2.75,3.25)/{01} , (4.25,3.25)/{05} , (5.75,3.25)/{13} , (7.25,3.25)/{09} ,
    (2.75,1.75)/{03} , (4.25,1.75)/{07} , (5.75,1.75)/{15} , (7.25,1.75)/{11} ,
    (2.75,0.25)/{02} , (4.25,0.25)/{06} , (5.75,0.25)/{14} , (7.25,0.25)/{10} }
  \draw \loc node{ \color{blue!90!black} \small{ $\txt$ }};
  
  % Draw the lines
  \draw
  % Finish drawing the grid
  [step=1.5cm,black,thin] (0,0) grid (7.5,7.5) % The Grid
  (0.0,7.5) -- (1.5,6.0) % Diagonal in the top left cell
  (1.5,6.10) -- (7.50,6.10) % Double line under top header row
  (1.40,0.0) -- (1.40,6.0) % Double line on left of header column one
  ;
  \end{tikzpicture}
  \caption{K-Map No.6}
  \label{}
\end{figure}

K-Map No.6 Minterm Expression:
\begin{align}
\label{}
\nonumber
&(A'BC'D)+(AB'C'D')+(AB'C'D)+(AB'CD')+(ABC'D)\\
\nonumber
&\int(A,B,C,D) = \sum(5,8,9,10,13)
\end{align}

K-Map No.6 Simplification:
\begin{align}
\nonumber
(BC'D)+(AB'C')+(AB'CD')
\end{align}

\subsection{K-Map Number 7}
\begin{figure}[H]
  \myfloatalign
  \begin{tikzpicture} [circuit logic US, scale=1.00]
  % make all path lines (the node shapes) a little thicker
  \tikzstyle{every path}=[line width=0.50mm]
  
  %********************************************************************
  % Adjust the settings below to display the 1's and rectangles
  %********************************************************************
  % Uncomment the appropriate lines below to insert ones where needed
  % Data Row 1
   \node[] at (2.25,5.25) {\Huge $ 1 $}; % 00
   \node[] at (3.75,5.25) {\Huge $ 1 $}; % 04
   \node[] at (5.25,5.25) {\Huge $ 1 $}; % 12
   \node[] at (6.75,5.25) {\Huge $ 1 $}; % 08
  % Data Row 2
   \node[] at (2.25,3.75) {\Huge $ 1 $}; % 01
   \node[] at (3.75,3.75) {\Huge $ 1 $}; % 05
   \node[] at (5.25,3.75) {\Huge $ 1 $}; % 13
   \node[] at (6.75,3.75) {\Huge $ 1 $}; % 09
  % Data Row 3
  % \node[] at (2.25,2.25) {\Huge $ 1 $}; % 03
  % \node[] at (3.75,2.25) {\Huge $ 1 $}; % 07
  % \node[] at (5.25,2.25) {\Huge $ 1 $}; % 15
  % \node[] at (6.75,2.25) {\Huge $ 1 $}; % 11
  % Data Row 4
  % \node[] at (2.25,0.75) {\Huge $ 1 $}; % 02
   \node[] at (3.75,0.75) {\Huge $ 1 $}; % 06
  % \node[] at (5.25,0.75) {\Huge $ 1 $}; % 14
  % \node[] at (6.75,0.75) {\Huge $ 1 $}; % 10
  
  % The coords for each cell - this is used to start the rectangle box
  \coordinate (cell00) at (1.50,4.50); \coordinate (cell01) at (1.50,3.00);
  \coordinate (cell02) at (1.50,0.00); \coordinate (cell03) at (1.50,1.50);
  \coordinate (cell04) at (3.00,4.50); \coordinate (cell05) at (3.00,3.00);
  \coordinate (cell06) at (3.00,0.00); \coordinate (cell07) at (3.00,1.50);
  \coordinate (cell08) at (6.00,4.50); \coordinate (cell09) at (6.00,3.00);
  \coordinate (cell10) at (6.00,0.00); \coordinate (cell11) at (6.00,1.50);
  \coordinate (cell12) at (4.50,4.50); \coordinate (cell13) at (4.50,3.00);
  \coordinate (cell14) at (4.50,0.00); \coordinate (cell15) at (4.50,1.50);
  
  % Set the ``at'' to the lower-left cell of the rectangle using the coords defined above
  % Set the minimum height/width to (number of cells) * 1.5. May have to decrease 
  % these by 0.1 to cut the rectangle just inside the cell.
  %  \node [draw,
  %  color=red!70!black,
  %    fill=red!20!white,
  %    fill opacity=0.3,
  %    minimum height=1.4cm,
  %    minimum width=3.0cm,
  %    double,
  %    rounded corners,
  %    anchor=south west] at (cell15) {};
  %  \node [draw,
  %    color=blue!70!black,
  %    fill=blue!20!white,
  %    fill opacity=0.3,
  %    minimum height=2.9cm,
  %    minimum width=1.5cm,
  %    double,
  %    rounded corners,
  %    anchor=south west] at (cell15) {};
  
  %********************************************************************
  % Shouldn't need to adjust anything below this point - this is just
  % the grid and the minterms.
  %********************************************************************	
  % Text in top-Left cell
  \node[] at (0.55,6.35) { $ \mathsf{ \mathbf{CD} } $ }; % CD
  \node[] at (1.05,7.05) { $ \mathsf{ \mathbf{AB} } $ }; % AB
  
  % Populate the top row header
  % In the following, the foreach lists a location/text pair
  % The the draw line draws the text at each location
  \foreach \loc/\txt in {(2.25,6.75)/{00},(3.75,6.75)/{01},(5.25,6.75)/{11},(6.75,6.75)/{10}}
  \draw \loc node{\Huge $\txt$};
  
  % Populate the header in column one
  \foreach \loc/\txt in {(0.75,5.25)/{00},(0.75,3.75)/{01},(0.75,2.25)/{11},(0.75,0.75)/{10}}
  \draw \loc node{\Huge $\txt$};
  
  % Populate the minterms
  \foreach \loc/\txt in { (2.75,4.75)/{00} , (4.25,4.75)/{04} , (5.75,4.75)/{12} , (7.25,4.75)/{08} ,
    (2.75,3.25)/{01} , (4.25,3.25)/{05} , (5.75,3.25)/{13} , (7.25,3.25)/{09} ,
    (2.75,1.75)/{03} , (4.25,1.75)/{07} , (5.75,1.75)/{15} , (7.25,1.75)/{11} ,
    (2.75,0.25)/{02} , (4.25,0.25)/{06} , (5.75,0.25)/{14} , (7.25,0.25)/{10} }
  \draw \loc node{ \color{blue!90!black} \small{ $\txt$ }};
  
  % Draw the lines
  \draw
  % Finish drawing the grid
  [step=1.5cm,black,thin] (0,0) grid (7.5,7.5) % The Grid
  (0.0,7.5) -- (1.5,6.0) % Diagonal in the top left cell
  (1.5,6.10) -- (7.50,6.10) % Double line under top header row
  (1.40,0.0) -- (1.40,6.0) % Double line on left of header column one
  ;
  \end{tikzpicture}
  \caption{K-Map No.7}
  \label{}
\end{figure}

K-Map No.7 Minterm Expression:
\begin{align}
\nonumber
&(A'B'C'D')+(A'B'C'D)+(A'BC'D')+(A'BC'D)(A'BCD')+(ABC'D')+(ABC'D)+(AB'C'D')+(AB'C'D)\\
\nonumber
&\int(A,B,C,D) = \sum(0,1,4,5,6,8,9,12,13)
\end{align}

K-Map No.7 Simplification:
\begin{align}
\nonumber
A'+(A'BCD')
\end{align}

\subsection{K-Map Number 8}
\begin{figure}[H]
  \myfloatalign
  \begin{tikzpicture} [circuit logic US, scale=1.00]
  % make all path lines (the node shapes) a little thicker
  \tikzstyle{every path}=[line width=0.50mm]
  
  %********************************************************************
  % Adjust the settings below to display the 1's and rectangles
  %********************************************************************
  % Uncomment the appropriate lines below to insert ones where needed
  % Data Row 1
  % \node[] at (2.25,5.25) {\Huge $ 1 $}; % 00
  % \node[] at (3.75,5.25) {\Huge $ 1 $}; % 04
  % \node[] at (5.25,5.25) {\Huge $ 1 $}; % 12
  % \node[] at (6.75,5.25) {\Huge $ 1 $}; % 08
  % Data Row 2
  % \node[] at (2.25,3.75) {\Huge $ 1 $}; % 01
  % \node[] at (3.75,3.75) {\Huge $ 1 $}; % 05
  % \node[] at (5.25,3.75) {\Huge $ 1 $}; % 13
   \node[] at (6.75,3.75) {\Huge $ 1 $}; % 09
  % Data Row 3
   \node[] at (2.25,2.25) {\Huge $ 1 $}; % 03
  % \node[] at (3.75,2.25) {\Huge $ 1 $}; % 07
  % \node[] at (5.25,2.25) {\Huge $ 1 $}; % 15
   \node[] at (6.75,2.25) {\Huge $ 1 $}; % 11
  % Data Row 4
  % \node[] at (2.25,0.75) {\Huge $ 1 $}; % 02
  % \node[] at (3.75,0.75) {\Huge $ 1 $}; % 06
  % \node[] at (5.25,0.75) {\Huge $ 1 $}; % 14
   \node[] at (6.75,0.75) {\Huge $ 1 $}; % 10
  
  % The coords for each cell - this is used to start the rectangle box
  \coordinate (cell00) at (1.50,4.50); \coordinate (cell01) at (1.50,3.00);
  \coordinate (cell02) at (1.50,0.00); \coordinate (cell03) at (1.50,1.50);
  \coordinate (cell04) at (3.00,4.50); \coordinate (cell05) at (3.00,3.00);
  \coordinate (cell06) at (3.00,0.00); \coordinate (cell07) at (3.00,1.50);
  \coordinate (cell08) at (6.00,4.50); \coordinate (cell09) at (6.00,3.00);
  \coordinate (cell10) at (6.00,0.00); \coordinate (cell11) at (6.00,1.50);
  \coordinate (cell12) at (4.50,4.50); \coordinate (cell13) at (4.50,3.00);
  \coordinate (cell14) at (4.50,0.00); \coordinate (cell15) at (4.50,1.50);
  
  % Set the ``at'' to the lower-left cell of the rectangle using the coords defined above
  % Set the minimum height/width to (number of cells) * 1.5. May have to decrease 
  % these by 0.1 to cut the rectangle just inside the cell.
  %  \node [draw,
  %  color=red!70!black,
  %    fill=red!20!white,
  %    fill opacity=0.3,
  %    minimum height=1.4cm,
  %    minimum width=3.0cm,
  %    double,
  %    rounded corners,
  %    anchor=south west] at (cell15) {};
  %  \node [draw,
  %    color=blue!70!black,
  %    fill=blue!20!white,
  %    fill opacity=0.3,
  %    minimum height=2.9cm,
  %    minimum width=1.5cm,
  %    double,
  %    rounded corners,
  %    anchor=south west] at (cell15) {};
  
  %********************************************************************
  % Shouldn't need to adjust anything below this point - this is just
  % the grid and the minterms.
  %********************************************************************	
  % Text in top-Left cell
  \node[] at (0.55,6.35) { $ \mathsf{ \mathbf{CD} } $ }; % CD
  \node[] at (1.05,7.05) { $ \mathsf{ \mathbf{AB} } $ }; % AB
  
  % Populate the top row header
  % In the following, the foreach lists a location/text pair
  % The the draw line draws the text at each location
  \foreach \loc/\txt in {(2.25,6.75)/{00},(3.75,6.75)/{01},(5.25,6.75)/{11},(6.75,6.75)/{10}}
  \draw \loc node{\Huge $\txt$};
  
  % Populate the header in column one
  \foreach \loc/\txt in {(0.75,5.25)/{00},(0.75,3.75)/{01},(0.75,2.25)/{11},(0.75,0.75)/{10}}
  \draw \loc node{\Huge $\txt$};
  
  % Populate the minterms
  \foreach \loc/\txt in { (2.75,4.75)/{00} , (4.25,4.75)/{04} , (5.75,4.75)/{12} , (7.25,4.75)/{08} ,
    (2.75,3.25)/{01} , (4.25,3.25)/{05} , (5.75,3.25)/{13} , (7.25,3.25)/{09} ,
    (2.75,1.75)/{03} , (4.25,1.75)/{07} , (5.75,1.75)/{15} , (7.25,1.75)/{11} ,
    (2.75,0.25)/{02} , (4.25,0.25)/{06} , (5.75,0.25)/{14} , (7.25,0.25)/{10} }
  \draw \loc node{ \color{blue!90!black} \small{ $\txt$ }};
  
  % Draw the lines
  \draw
  % Finish drawing the grid
  [step=1.5cm,black,thin] (0,0) grid (7.5,7.5) % The Grid
  (0.0,7.5) -- (1.5,6.0) % Diagonal in the top left cell
  (1.5,6.10) -- (7.50,6.10) % Double line under top header row
  (1.40,0.0) -- (1.40,6.0) % Double line on left of header column one
  ;
  \end{tikzpicture}
  \caption{K-Map No.8}
  \label{}
\end{figure}

K-Map No.8 Minterm Expression:
\begin{align}
\nonumber
&(A'B'CD)+(AB'C'D)+(AB'CD')+(AB'CD)\\
\nonumber
&\int(A,B,C,D) = \sum(3,9,10,11)
\end{align}

K-Map No.8 Simplification:
\begin{align}
\nonumber
(B'CD)+(AB'D)+(AB'C)
\end{align}

\subsection{K-Map Number 9}
\begin{figure}[H]
  \myfloatalign
  \begin{tikzpicture} [circuit logic US, scale=1.00]
  % make all path lines (the node shapes) a little thicker
  \tikzstyle{every path}=[line width=0.50mm]
  
  %********************************************************************
  % Adjust the settings below to display the 1's and rectangles
  %********************************************************************
  % Uncomment the appropriate lines below to insert ones where needed
  % Data Row 1
  % \node[] at (2.25,5.25) {\Huge $ 1 $}; % 00
  % \node[] at (3.75,5.25) {\Huge $ 1 $}; % 04
   \node[] at (5.25,5.25) {\Huge $ 1 $}; % 12
   \node[] at (6.75,5.25) {\Huge $ 1 $}; % 08
  % Data Row 2
  % \node[] at (2.25,3.75) {\Huge $ 1 $}; % 01
  % \node[] at (3.75,3.75) {\Huge $ 1 $}; % 05
   \node[] at (5.25,3.75) {\Huge $ 1 $}; % 13
   \node[] at (6.75,3.75) {\Huge $ 1 $}; % 09
  % Data Row 3
  % \node[] at (2.25,2.25) {\Huge $ 1 $}; % 03
  % \node[] at (3.75,2.25) {\Huge $ 1 $}; % 07
   \node[] at (5.25,2.25) {\Huge $ 1 $}; % 15
  % \node[] at (6.75,2.25) {\Huge $ 1 $}; % 11
  % Data Row 4
  % \node[] at (2.25,0.75) {\Huge $ 1 $}; % 02
  % \node[] at (3.75,0.75) {\Huge $ 1 $}; % 06
  % \node[] at (5.25,0.75) {\Huge $ 1 $}; % 14
  % \node[] at (6.75,0.75) {\Huge $ 1 $}; % 10
  
  % The coords for each cell - this is used to start the rectangle box
  \coordinate (cell00) at (1.50,4.50); \coordinate (cell01) at (1.50,3.00);
  \coordinate (cell02) at (1.50,0.00); \coordinate (cell03) at (1.50,1.50);
  \coordinate (cell04) at (3.00,4.50); \coordinate (cell05) at (3.00,3.00);
  \coordinate (cell06) at (3.00,0.00); \coordinate (cell07) at (3.00,1.50);
  \coordinate (cell08) at (6.00,4.50); \coordinate (cell09) at (6.00,3.00);
  \coordinate (cell10) at (6.00,0.00); \coordinate (cell11) at (6.00,1.50);
  \coordinate (cell12) at (4.50,4.50); \coordinate (cell13) at (4.50,3.00);
  \coordinate (cell14) at (4.50,0.00); \coordinate (cell15) at (4.50,1.50);
  
  % Set the ``at'' to the lower-left cell of the rectangle using the coords defined above
  % Set the minimum height/width to (number of cells) * 1.5. May have to decrease 
  % these by 0.1 to cut the rectangle just inside the cell.
  %  \node [draw,
  %  color=red!70!black,
  %    fill=red!20!white,
  %    fill opacity=0.3,
  %    minimum height=1.4cm,
  %    minimum width=3.0cm,
  %    double,
  %    rounded corners,
  %    anchor=south west] at (cell15) {};
  %  \node [draw,
  %    color=blue!70!black,
  %    fill=blue!20!white,
  %    fill opacity=0.3,
  %    minimum height=2.9cm,
  %    minimum width=1.5cm,
  %    double,
  %    rounded corners,
  %    anchor=south west] at (cell15) {};
  
  %********************************************************************
  % Shouldn't need to adjust anything below this point - this is just
  % the grid and the minterms.
  %********************************************************************	
  % Text in top-Left cell
  \node[] at (0.55,6.35) { $ \mathsf{ \mathbf{CD} } $ }; % CD
  \node[] at (1.05,7.05) { $ \mathsf{ \mathbf{AB} } $ }; % AB
  
  % Populate the top row header
  % In the following, the foreach lists a location/text pair
  % The the draw line draws the text at each location
  \foreach \loc/\txt in {(2.25,6.75)/{00},(3.75,6.75)/{01},(5.25,6.75)/{11},(6.75,6.75)/{10}}
  \draw \loc node{\Huge $\txt$};
  
  % Populate the header in column one
  \foreach \loc/\txt in {(0.75,5.25)/{00},(0.75,3.75)/{01},(0.75,2.25)/{11},(0.75,0.75)/{10}}
  \draw \loc node{\Huge $\txt$};
  
  % Populate the minterms
  \foreach \loc/\txt in { (2.75,4.75)/{00} , (4.25,4.75)/{04} , (5.75,4.75)/{12} , (7.25,4.75)/{08} ,
    (2.75,3.25)/{01} , (4.25,3.25)/{05} , (5.75,3.25)/{13} , (7.25,3.25)/{09} ,
    (2.75,1.75)/{03} , (4.25,1.75)/{07} , (5.75,1.75)/{15} , (7.25,1.75)/{11} ,
    (2.75,0.25)/{02} , (4.25,0.25)/{06} , (5.75,0.25)/{14} , (7.25,0.25)/{10} }
  \draw \loc node{ \color{blue!90!black} \small{ $\txt$ }};
  
  % Draw the lines
  \draw
  % Finish drawing the grid
  [step=1.5cm,black,thin] (0,0) grid (7.5,7.5) % The Grid
  (0.0,7.5) -- (1.5,6.0) % Diagonal in the top left cell
  (1.5,6.10) -- (7.50,6.10) % Double line under top header row
  (1.40,0.0) -- (1.40,6.0) % Double line on left of header column one
  ;
  \end{tikzpicture}
  \caption{K-Map No.9}
  \label{}
\end{figure}

K-Map No.9 Minterm Expression:
\begin{align}
\nonumber
&(AB'C'D')+(AB'C'D)+(ABC'D')+(ABC'D)+(ABCD)\\
\nonumber
&\int(A,B,C,D) = \sum(8,9,12,13,15)
\end{align}

K-Map No.9 Simplification:
\begin{align}
\nonumber
(AC')+(ABD)
\end{align}

\subsection{K-Map Number 10}
\begin{figure}[H]
  \myfloatalign
  \begin{tikzpicture} [circuit logic US, scale=1.00]
  % make all path lines (the node shapes) a little thicker
  \tikzstyle{every path}=[line width=0.50mm]
  
  %********************************************************************
  % Adjust the settings below to display the 1's and rectangles
  %********************************************************************
  % Uncomment the appropriate lines below to insert ones where needed
  % Data Row 1
  % \node[] at (2.25,5.25) {\Huge $ 1 $}; % 00
  % \node[] at (3.75,5.25) {\Huge $ 1 $}; % 04
   \node[] at (5.25,5.25) {\Huge $ 1 $}; % 12
  % \node[] at (6.75,5.25) {\Huge $ 1 $}; % 08
  % Data Row 2
  % \node[] at (2.25,3.75) {\Huge $ 1 $}; % 01
  % \node[] at (3.75,3.75) {\Huge $ 1 $}; % 05
  % \node[] at (5.25,3.75) {\Huge $ 1 $}; % 13
  % \node[] at (6.75,3.75) {\Huge $ 1 $}; % 09
  % Data Row 3
   \node[] at (2.25,2.25) {\Huge $ 1 $}; % 03
  % \node[] at (3.75,2.25) {\Huge $ 1 $}; % 07
   \node[] at (5.25,2.25) {\Huge $ 1 $}; % 15
   \node[] at (6.75,2.25) {\Huge $ 1 $}; % 11
  % Data Row 4
  % \node[] at (2.25,0.75) {\Huge $ 1 $}; % 02
  % \node[] at (3.75,0.75) {\Huge $ 1 $}; % 06
   \node[] at (5.25,0.75) {\Huge $ 1 $}; % 14
  % \node[] at (6.75,0.75) {\Huge $ 1 $}; % 10
  
  % The coords for each cell - this is used to start the rectangle box
  \coordinate (cell00) at (1.50,4.50); \coordinate (cell01) at (1.50,3.00);
  \coordinate (cell02) at (1.50,0.00); \coordinate (cell03) at (1.50,1.50);
  \coordinate (cell04) at (3.00,4.50); \coordinate (cell05) at (3.00,3.00);
  \coordinate (cell06) at (3.00,0.00); \coordinate (cell07) at (3.00,1.50);
  \coordinate (cell08) at (6.00,4.50); \coordinate (cell09) at (6.00,3.00);
  \coordinate (cell10) at (6.00,0.00); \coordinate (cell11) at (6.00,1.50);
  \coordinate (cell12) at (4.50,4.50); \coordinate (cell13) at (4.50,3.00);
  \coordinate (cell14) at (4.50,0.00); \coordinate (cell15) at (4.50,1.50);
  
  % Set the ``at'' to the lower-left cell of the rectangle using the coords defined above
  % Set the minimum height/width to (number of cells) * 1.5. May have to decrease 
  % these by 0.1 to cut the rectangle just inside the cell.
  %  \node [draw,
  %  color=red!70!black,
  %    fill=red!20!white,
  %    fill opacity=0.3,
  %    minimum height=1.4cm,
  %    minimum width=3.0cm,
  %    double,
  %    rounded corners,
  %    anchor=south west] at (cell15) {};
  %  \node [draw,
  %    color=blue!70!black,
  %    fill=blue!20!white,
  %    fill opacity=0.3,
  %    minimum height=2.9cm,
  %    minimum width=1.5cm,
  %    double,
  %    rounded corners,
  %    anchor=south west] at (cell15) {};
  
  %********************************************************************
  % Shouldn't need to adjust anything below this point - this is just
  % the grid and the minterms.
  %********************************************************************	
  % Text in top-Left cell
  \node[] at (0.55,6.35) { $ \mathsf{ \mathbf{CD} } $ }; % CD
  \node[] at (1.05,7.05) { $ \mathsf{ \mathbf{AB} } $ }; % AB
  
  % Populate the top row header
  % In the following, the foreach lists a location/text pair
  % The the draw line draws the text at each location
  \foreach \loc/\txt in {(2.25,6.75)/{00},(3.75,6.75)/{01},(5.25,6.75)/{11},(6.75,6.75)/{10}}
  \draw \loc node{\Huge $\txt$};
  
  % Populate the header in column one
  \foreach \loc/\txt in {(0.75,5.25)/{00},(0.75,3.75)/{01},(0.75,2.25)/{11},(0.75,0.75)/{10}}
  \draw \loc node{\Huge $\txt$};
  
  % Populate the minterms
  \foreach \loc/\txt in { (2.75,4.75)/{00} , (4.25,4.75)/{04} , (5.75,4.75)/{12} , (7.25,4.75)/{08} ,
    (2.75,3.25)/{01} , (4.25,3.25)/{05} , (5.75,3.25)/{13} , (7.25,3.25)/{09} ,
    (2.75,1.75)/{03} , (4.25,1.75)/{07} , (5.75,1.75)/{15} , (7.25,1.75)/{11} ,
    (2.75,0.25)/{02} , (4.25,0.25)/{06} , (5.75,0.25)/{14} , (7.25,0.25)/{10} }
  \draw \loc node{ \color{blue!90!black} \small{ $\txt$ }};
  
  % Draw the lines
  \draw
  % Finish drawing the grid
  [step=1.5cm,black,thin] (0,0) grid (7.5,7.5) % The Grid
  (0.0,7.5) -- (1.5,6.0) % Diagonal in the top left cell
  (1.5,6.10) -- (7.50,6.10) % Double line under top header row
  (1.40,0.0) -- (1.40,6.0) % Double line on left of header column one
  ;
  \end{tikzpicture}
  \caption{K-Map No.10}
  \label{}
\end{figure}

K-Map No.10 Minterm Expression:
\begin{align}
\nonumber
&(A'B'CD)+(AB'CD)+(ABC'D')+(ABCD')+(ABCD)\\
\nonumber
&\int(A,B,C,D) = \sum(3,11,12,14,15)
\end{align}

K-Map No.10 Simplification:
\begin{align}
\nonumber
(B'CD)+(ABD')+(ABC)
\end{align}


\section{Karma}

\subsection{Problem Number 1}
\begin{align}
\nonumber
&\int(A,B,C,D,E) = \sum(0,4,12,16,20,28) \\
\nonumber
&simplified = (B'D'E')+(CD'E')
\end{align}

\subsection{Problem Number 2}
\begin{align}
\nonumber
&\int(A,B,C,D,E) = \sum(5,7,13,15,18,19,25) \\
\nonumber
&simplified = (A'CE)+(AB'C'D)+(ABC'D'E)
\end{align}

\subsection{Problem Number 3}
\begin{align}
\nonumber
&\int(A,B,C,D,E) = \sum(1,3,19,23,28,31) \\
\nonumber
&simplified = (AB'DE)+(ACDE)+(ABCD'E')+(A'B'C'E)
\end{align}

\subsection{Problem Number 4}
\begin{align}
\nonumber
&\int(A,B,C,D,E) = \sum(11,13,15,16,17,19,31) \\
\nonumber
&simplified = (AB'C'E)+(AB'C'D')+(A'BCE)+(A'BDE)+(BCDE)
\end{align}

\subsection{Problem Number 5}
\begin{align}
\nonumber
&\int(A,B,C,D,E) = \sum(0,2,4,10,15,19,28,29,31) \\
\nonumber
&simplified = (A'B'D'E')+(AB'C'DE)+(ABCD')+(BCDE)+(A'C'DE')
\end{align}

\subsection{Problem Number 6}
\begin{align}
\nonumber
&\int(A,B,C,D,E) = \sum(10,11,14,15,26,27,30,31) \\
\nonumber
&simplified = (BD)
\end{align}

\subsection{Problem Number 7}
\begin{align}
\nonumber
&\int(A,B,C,D,E) = \sum(5,6,7,20,21) \\
\nonumber
&simplified = (A'B'CE) + (AB'CD') + (A'B'CD)
\end{align}

\subsection{Problem Number 8}
\begin{align}
\nonumber
&\int(A,B,C,D,E) = \sum(5,9,17,29) \\
\nonumber
&simplified = (ABCD'E) + (AB'C'D'E) + (A'BC'D'E) + (A'B'CD'E)
\end{align}

\subsection{Problem Number 9}
\begin{align}
\nonumber
&\int(A,B,C,D,E) = \sum(5,7,8,10,21,23,24,26) \\
\nonumber
&simplified = (B'CE) + (BC'E')
\end{align}

\subsection{Problem Number 10}
\begin{align}
\nonumber
&\int(A,B,C,D,E) = \sum(9,13,15,21,29,31) \\
\nonumber
&simplified = (ACD'E) + (A'BD'E) + (BCE)
\end{align}

\subsection{Problem Number 11}
\begin{align}
\nonumber
&\int(A,B,C,D,E) = \sum(4,6,7,12,20,21,23) \\
\nonumber
&simplified = (A'CD'E') + (B'CDE) + (A'B'CE') + (AB'CD')
\end{align}

\subsection{Problem Number 12}
\begin{align}
\nonumber
&\int(A,B,C,D,E) = \sum(1,3,5,25,27,29) \\
\nonumber
&simplified = (A'B'D'E) + (ABD'E) + (A'B'C'E) + (ABC'E)
\end{align}

\subsection{Problem Number 13}
\begin{align}
\nonumber
&\int(A,B,C,D,E) = \sum(0,2,8,10,16,18,24,26) \\
\nonumber
&simplified = (C'E')
\end{align}

\subsection{Problem Number 14}
\begin{align}
\nonumber
&\int(A,B,C,D,E) = \sum(8,9,10,11,12,13,14,15,24,25,26,27,28,29,30,31) \\
\nonumber
&simplified = (B)
\end{align}

\subsection{Problem Number 15}
\begin{align}
\nonumber
&\int(A,B,C,D,E) = \sum(4,12,13,15,21,23,29,31) \\
\nonumber
&simplified = (A'CD'E') + (ACE) + (BCE)
\end{align}



\section{End}











